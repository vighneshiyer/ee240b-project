\documentclass{article}
\usepackage{tikz}
\usepackage{graphicx}
\usepackage{float}
\usepackage{amsmath}
\usepackage{circuitikz}

\begin{document}
\section{Here}

% coordinates start in the bottom left corner (0, 0) is bottom left
% \draw (coordinate) -- (coordinate) -- (coordinate) | draws lines from each coordinate to the next
% \draw (coordinate) -- (c) -- (c) -- cycle | loops back to the originally drawn coordinate
% \draw (coordinate) rectangle (coordinate) | draws a rect
% \draw (0,0) .. controls (0,5) and (5,0) .. (4,4) | draws a line from (0,0) to (4,4) with control points which 'influence' the line
\begin{figure}[H]
\begin{circuitikz}[line width=1.25pt]
  \draw
  (0, 0) node[gm amp] (otan) {}
  (otan.+) to[short,-o] ++(-1,0) node[anchor=east] {$V_{in}^-$}
    (otan.out) to[short] ++(0.5,0) to[short] ++(0,1.5) coordinate (otanfb)
    to[short] ++(-1,0) to[R,l_=$R$] ++(-1.5,0)
    to[short] ++(-1,0) coordinate (voutp)
    (voutp) to[short] ++(0,-1) -- (otan.-)
    (otan.-) to[short,-o] ++(-1,0) node[anchor=east] {$V_{out}^+$}
    (otanfb) to[C,l_=$C$] ++(0,1.5) coordinate (cap1top)
    (voutp) to[C,l^=$C$] ++(0,1.5) coordinate (cap2top)
    (cap1top) to[short] ++(-1,0) to[R,l_=$R$] ++(-1.5,0) -- (cap2top)
    (0, 5) node[gm amp,yscale=-1] (otap) {}
    (otap.out) to[short] ++(0.5,0) -- (cap1top)
    (cap2top) to[short] ++(0,1.5) -- (otap.-)
    (otap.-) to[short,-o] ++(-1,0) node[anchor=east] {$V_{out}^-$}
    (otap.+) to[short,-o] ++(-1,0) node[anchor=east] {$V_{in}^+$}
;\end{circuitikz}
\end{figure}

\newpage
\begin{figure}[H]
  \input{ota_schematic.tex}
\end{figure}

\newpage
\begin{figure}[H]
  \ctikzset{tripoles/pmos style/emptycircle}
\ctikzset{tripoles/mos style/arrows}
\begin{circuitikz}[line width=1pt]
    \draw
    (0,0) node[nmos](tail) {}
    (-6,0) node[nmos,xscale=-1](tailb) {}
    (0,2) node[nmos](tailcas) {}
    (-6,2) node[nmos,xscale=-1](tailcasb) {}
    (-2,4) node[nmos](inp) {}
    (2,4) node[nmos,xscale=-1](inn) {}
    (-2,6) node[pmos,xscale=-1](casp) {}
    (2,6) node[pmos](casn) {}
    (-2,8) node[pmos,xscale=-1](mirp) {}
    (2,8) node[pmos](mirn) {}
    (0,9.5) node[anchor=north] {$V_{DD}$}
    (4,1) node[op amp](cmamp) {}
    (mirn.S) -- (mirp.S)
    (mirp.G) to[short] ++(0.5,0) to[short] ++(0,-3) to[short,-*] ++(-1.5,0)
    (mirn.G) to[short] ++(-0.5,0) to[short] ++(0,-3) to[short,-*] ++(1.5,0)
    (mirp.D) -- (casp.S)
    (mirn.D) -- (casn.S)
    (casp.G) -- (casn.G)
    (casp.G) to[short] ++(1,0) to[short,-o] ++(0,-1) node[anchor=north] {$V_{b,cas,p}$}
    (casp.D) -- (inp.D)
    (casn.D) -- (inn.D)
    (casp.D) to[short] ++(0,-0.5) to[short,-o] ++(-1,0) node[anchor=east] {$V_{out}^-$}
    (casn.D) to[short] ++(0,-0.5) to[short,-o] ++(1,0) node[anchor=west] {$V_{out}^+$}
    (inp.G) to[short,-o] ++(0,0) node[anchor=east] {$V_{in}^+$}
    (inn.G) to[short,-o] ++(0,0) node[anchor=west] {$V_{in}^-$}
    (inn.S) -- (inp.S)
    (tailcas.D) to[short,-*] ++(0,0.45)
    (tailcas.G) -- (tailcasb.G)
    (tailcasb.D) to[short,*-] ++(2,0) to[short,-*] ++(0,-2.75) node[anchor=north] {$V_{bias}$} ++(0,0)
    (tailb.G) to[short] ++(1,0)
    (tail.D) -- (tailcas.S)
    (tailb.D) -- (tailcasb.S)
    (tail.S) node[ground] {} ++(0,0)
    (tailb.S) node[ground] {} ++(0,0)
    (tail.G) to[short,-o] ++(0,0) node[anchor=east] {$V_{cmfb}$}
    (tailcas.G) ++(-1.5,0) to[short,-o] ++(0,-0.5) node[anchor=north] {$V_{b,cas,n}$}
    (tailcasb.D) to[ioosource] ++(0,2) coordinate(ibias)
    (ibias) ++(0.5,-1) node[anchor=west] {$I_{bias}$}
    (ibias) node[rground,yscale=-1] {} ++(0,0) to[open] ++(0,0.5) node[anchor=south] {$V_{DD}$}
    (cmamp.+) to[short,-o] ++(0,0) node[anchor=east] {$V_{cm,sense}$}
    (cmamp.-) to[short,-o] ++(0,0) node[anchor=east] {$V_{cm,ref}$}
    (cmamp.out) ++(1,0) node[adder](biasadder) {} ++(0,0)
    (cmamp.out) -- (biasadder.1)
    (biasadder.2) to[short,-o] ++(0,-1) node[anchor=north] {$V_{bias}$}
    (biasadder.3) to[short,-o] ++(1,0)  node[anchor=west] {$V_{cmfb}$} ++(0,0)
    (mirp.G) ++(-1,0) node[anchor=east] {$\frac{12W}{4L}$}
    (mirn.G) ++(1,0) node[anchor=west] {$\frac{12W}{4L}$}
    (casp.G) ++(-1,0) node[anchor=east] {$\frac{6W}{45 \text{nm}}$}
    (casn.G) ++(1,0) node[anchor=west] {$\frac{6W}{45 \text{nm}}$}
    (inp.G) ++(1,0) node[anchor=west] {$\frac{6W}{L}$}
    (inn.G) ++(-1,0) node[anchor=east] {$\frac{6W}{L}$}
    (tailcas.G) ++(1,0) node[anchor=west] {$\frac{48W}{45 \text{nm}}$}
    (tail.G) ++(1,0) node[anchor=west] {$\frac{48W}{2L}$}
    (tailcasb.G) ++(-1,0) node[anchor=east] {$\frac{6W}{45 \text{nm}}$}
    (tailb.G) ++(-1,0) node[anchor=east] {$\frac{6W}{2L}$}
;\end{circuitikz}

\end{figure}

\newpage
\begin{figure}[H]
  \ctikzset{tripoles/pmos style/emptycircle}
\ctikzset{tripoles/mos style/arrows}
\begin{circuitikz}[line width=1pt]
    \draw
    (0,0) node[nmos](ntail) {}
    (-2,2) node[nmos](ninp) {}
    (2,2) node[nmos,xscale=-1](ninn) {}
    (-2,4) node[nmos,xscale=-1](ncasp) {}
    (2,4) node[nmos](ncasn) {}
    (-2,6) node[pmos,xscale=-1](pcasp) {}
    (2,6) node[pmos](pcasn) {}
    (-2,8) node[pmos,xscale=-1](pmirp) {}
    (2,8) node[pmos](pmirn) {}
    (0,9.5) node[anchor=north] {$V_{DD}$}
    (ntail.S) to[short] node[ground] {} ++(0,0)
    (ntail.D) to[short] ++(-2,0) -- (ninp.S)
    (ntail.D) to[short] ++(2,0) -- (ninn.S)
    (ncasp.G) -- (ncasn.G)
    (pmirp.G) -- (pmirn.G)
    (ncasp.D) to[short,-*] ++(0,0.25) to[short] ++(2,0) to[short,-*] ++(0,3)
    (ncasp.D) -- (pcasp.D)
    (pcasp.S) -- (pmirp.D)
    (pcasn.S) -- (pmirn.D)
    (pmirp.S) -- (pmirn.S)
    (pcasn.D) -- (ncasn.D)
    (ninn.D) -- (ncasn.S)
    (ninp.D) -- (ncasp.S)
    (pcasn.G) -- (pcasp.G)
    (pcasp.G) to[short,-o] ++(-2,0) node[anchor=east] {$V_{b,cas,p}$}
    (ncasp.G) to[short,-o] ++(-2,0) node[anchor=east] {$V_{b,cas,n}$}
    (ntail.G) to[short,-o] ++(-0.5,0) node[anchor=east] {$V_{b,tail}$}
    (pmirp.G) ++(-1,0) node[anchor=east] {$\frac{12W}{4L}$}
    (pmirn.G) ++(1,0) node[anchor=west] {$\frac{12W}{4L}$}
    (pcasp.G) ++(0.1,0.5) node {$\frac{6W}{45 \text{nm}}$}
    (pcasn.G) ++(-0.1,0.5) node {$\frac{6W}{45 \text{nm}}$}
    (ncasn.G) ++(0.1,-0.6) node {$\frac{6W}{45 \text{nm}}$}
    (ncasp.G) ++(-0.1,-0.6) node {$\frac{6W}{45 \text{nm}}$}
    (ninp.G) ++(1.3,0) node {$\frac{6W}{L}$}
    (ninn.G) ++(-1.3,0) node {$\frac{6W}{L}$}
    (ntail.G) ++(1.3,0) node {$\frac{48W}{2L}$}
    (ncasn.D) ++(0,0.25) to[short,-o] ++(1.5,0) node[anchor=west] {$V_{cm,ref}$}
    (ninp.G) to[short] ++(0,1) to[short] ++(6,0)
    (ninn.G) to[short,-*] ++(0,1) to[short] ++(0,2)
;\end{circuitikz}

\end{figure}

\newpage
\begin{figure}
  \centering
  \centering
\ctikzset{tripoles/pmos style/emptycircle}
\ctikzset{tripoles/mos style/arrows}
\begin{circuitikz}[line width=1pt]
    %\draw[step=1cm,gray,very thin] (-6,-6) grid (12,12);
    \draw
    (0,0) node[vsourcesinshape, rotate=90](vinac) {}
    (-1,-1) to[american voltage source] ++(-1,0) to[short] ++(-1,0) node[ground] {} ++(-1,0)
    (-1,1) to[american voltage source] ++(-1,0) to[short] ++(-1,0) node[ground] {} ++(-1,0)
    (0,0.5) to[short] ++(0,0.5) to[short] ++(-1,0)
    (0,-0.5) to[short] ++(0,-0.5) to[short] ++(-1,0)
    (1,-2) rectangle (4,2) node[pos=0.5] {CM Reject Stage}
    (0,1) to[short,-*] ++(1,0) node[anchor=west] {$+$}
    (0,-1) to[short,-*] ++(1,0) node[anchor=west] {$-$}
    (5,-2) rectangle (8,2) node[pos=0.5] {OTA Filter Stage}
    (4,1) node[anchor=east] {$-$} to[short,*-*] ++(1,0) node[anchor=west] {$+$}
    (4,-1) node[anchor=east] {$+$} to[short,*-*] ++(1,0) node[anchor=west] {$-$}
    (9,-2) rectangle (12,2) node[pos=0.5] {OTA Filter Stage}
    (8,1) node[anchor=east] {$-$} to[short,*-*] ++(1,0) node[anchor=west] {$+$}
    (8,-1) node[anchor=east] {$+$} to[short,*-*] ++(1,0) node[anchor=west] {$-$}
    (12,1) node[anchor=east] {$-$} to[short,*-o] ++(1,0) node[anchor=west] {$+$}
    (12,-1) node[anchor=east] {$+$} to[short,*-o] ++(1,0) node[anchor=west] {$-$}
    (0,-3) node[rground,yscale=-1](vdd) {}
    (0,-3) to[ioosource] ++(0,-1) {}
    (0,-6) node[nmos,xscale=-1](mos) {} ++(0,-1)
    (mos.G) to[short] ++(0,1) to[short,-*] ++(-1,0)
    (vdd.1) ++(0,-1) -- (mos.D)
    (mos.G) ++(-1,0) node[anchor=east] {$\frac{6W}{2L}$}
    (mos.S) ++(0,0) node[ground] {}
    (mos.G) to[short,-o] ++(1,0) node[anchor=west] {$V_{b,tail}$}
    (4,-4) rectangle (6,-6) node[pos=0.5] {Replica}
    (5,-4) to[short,-o] ++(0,1) node[anchor=south] {$V_{cm,ref}$}
    (vinac) ++(-0.5,0) node[anchor=east] {$V_{i}$}
    (vinac) ++(-1.5,1.5) node[anchor=south] {$V_{cm,in}$}
    (vinac) ++(-1.5,-1.5) node[anchor=north] {$V_{cm,in}$}
;\end{circuitikz}

\end{figure}

\end{document}

